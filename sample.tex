%%%%%%%%%%%%%%%%%
% This is an sample CV template created using altacv.cls
% (v1.6.4, 13 Nov 2021) written by LianTze Lim (liantze@gmail.com). Now compiles with pdfLaTeX, XeLaTeX and LuaLaTeX.
%
%% It may be distributed and/or modified under the
%% conditions of the LaTeX Project Public License, either version 1.3
%% of this license or (at your option) any later version.
%% The latest version of this license is in
%%    http://www.latex-project.org/lppl.txt
%% and version 1.3 or later is part of all distributions of LaTeX
%% version 2003/12/01 or later.
%%%%%%%%%%%%%%%%

%% Use the "normalphoto" option if you want a normal photo instead of cropped to a circle
% \documentclass[10pt,a4paper,normalphoto]{altacv}

\documentclass[10pt,a4paper,ragged2e,withhyper]{altacv}
%% AltaCV uses the fontawesome5 and packages.
%% See http://texdoc.net/pkg/fontawesome5 for full list of symbols.

% Change the page layout if you need to
\geometry{left=1.25cm,right=1.25cm,top=1.5cm,bottom=1.5cm,columnsep=1.2cm}

% The paracol package lets you typeset columns of text in parallel
\usepackage{paracol}
\usepackage{ragged2e}
\usepackage{hyperref}

% Change the font if you want to, depending on whether
% you're using pdflatex or xelatex/lualatex
\ifxetexorluatex
  % If using xelatex or lualatex:
  \setmainfont{Roboto Slab}
  \setsansfont{Lato}
  \renewcommand{\familydefault}{\sfdefault}
\else
  % If using pdflatex:
  \usepackage[rm]{roboto}
  \usepackage[defaultsans]{lato}
  % \usepackage{sourcesanspro}
  \renewcommand{\familydefault}{\sfdefault}
\fi

% Change the colours if you want to
\definecolor{SlateGrey}{HTML}{2E2E2E}
\definecolor{LightGrey}{HTML}{666666}
\definecolor{DarkPastelRed}{HTML}{AB000D}
\definecolor{Red}{HTML}{E53935}
\definecolor{GoldenEarth}{HTML}{E7D192}
\colorlet{name}{black}
\colorlet{tagline}{Red}
\colorlet{heading}{DarkPastelRed}
\colorlet{headingrule}{DarkPastelRed}
\colorlet{subheading}{Red}
\colorlet{accent}{Red}
\colorlet{emphasis}{SlateGrey}
\colorlet{body}{LightGrey}

% Change some fonts, if necessary
\renewcommand{\namefont}{\Huge\rmfamily\bfseries}
%%\renewcommand{\personalinfofont}{\footnotesize}
\renewcommand{\cvsectionfont}{\LARGE\rmfamily\bfseries}
\renewcommand{\cvsubsectionfont}{\large\bfseries}


% Change the bullets for itemize and rating marker
% for \cvskill if you want to
\renewcommand{\itemmarker}{{\small\textbullet}}
\renewcommand{\ratingmarker}{\faCircle}

%% Use (and optionally edit if necessary) this .tex if you
%% want to use an author-year reference style like APA(6)
%% for your publication list
\input{pubs-authoryear}

%% Use (and optionally edit if necessary) this .tex if you
%% want an originally numerical reference style like IEEE
%% for your publication list
% \input{pubs-num}

\begin{document}
\name{MATTEO FACCI}

% \photoL{2.5cm}{Yacht_High,Suitcase_High}

\personalinfo{%
  % Not all of these are required!
  \email{facci.matteo.job@gmail.com}
  \location{Padua, Italy}
  \linkedin{Matteo-Facci}
  \github{faccimatteo}
  %% You can add your own arbitrary detail with
  %% \printinfo{symbol}{detail}[optional hyperlink prefix]
  % \printinfo{\faPaw}{Hey ho!}[https://example.com/]
  %% Or you can declare your own field with
  %% \NewInfoFiled{fieldname}{symbol}[optional hyperlink prefix] and use it:
  % \NewInfoField{gitlab}{\faGitlab}[https://gitlab.com/]
  % \gitlab{your_id}
  %%
  %% For services and platforms like Mastodon where there isn't a
  %% straightforward relation between the user ID/nickname and the hyperlink,
  %% you can use \printinfo directly e.g.
  % \printinfo{\faMastodon}{@username@instace}[https://instance.url/@username]
  %% But if you absolutely want to create new dedicated info fields for
  %% such platforms, then use \NewInfoField* with a star:
  % \NewInfoField*{mastodon}{\faMastodon}
  %% then you can use \mastodon, with TWO arguments where the 2nd argument is
  %% the full hyperlink.
  % \mastodon{@username@instance}{https://instance.url/@username}
}
\makecvheader
%% Depending on your tastes, you may want to make fonts of itemize environments slightly smaller
% \AtBeginEnvironment{itemize}{\small}

%% Set the left/right column width ratio to 6:4.
\columnratio{1}

% Start a 2-column paracol. Both the left and right columns will automatically
% break across pages if things get too long.
\begin{paracol}{2}
\cvsection{Experience}

\cvevent{R\&D Engineer}{Nethive}{March 2022 - Present}{Padua, Italy}
\begin{itemize}
\item Working on the backend side of microservice-based privacy solutions projects and event-driven applications.
\item Deploying cloud-based applications with particular attention to the security of the latter.
\end{itemize}
\medskip

\cvevent{High school intern}{Zucchetti}{May 2017 - June 2017}{Padua, Italy}
\par
During this two-month internship, I have been beside different company departments such as Helpdesk, Marketing, and Soc. In particular, I mainly stayed in the Development department, working on implementing some features for the Intranet management software.

\begin{flushleft}
    \cvsection{Education}
\end{flushleft}

\cvevent{M.Sc.\ in Software Dependability and Cyber Security}{Ca' Foscari University of Venice}{2021 -- in progress}{Venice, Italy}

\cvevent{B.Sc.\ in Computer Science}{Ca' Foscari University of Venice}{2018 -- 2021}{Venice, Italy}
\begin{itemize}
    \item Thesis: "Implementation of a Netdisco-based microservice to search for hidden devices on a network".
    \item Grade: 100/110
\end{itemize}

\begin{flushleft}
    \cvsection{Side Projects}
\end{flushleft}


\cvevent{CPU Virtualizator}{\cvtag{C}\cvtag{Bash}}{}{}
\faGithub\href{https://github.com/faccimatteo/CPU-Virtualizator}{CPU Virtualizator}
\par
C project that simulate CPU execution using smaller assebly-like instructions.
\medskip

\cvevent{Connect4}{\cvtag{MongoDB}\cvtag{Express}\cvtag{Angular}\cvtag{Node.js}\cvtag{Web Socket}\cvtag{Docker}}{}{}
\faGithub\href{https://github.com/faccimatteo/Connect4}{Connect4}
\par

MEAN stack web application that allows you to play and spectate Connect 4 games with your friends and random players.
\medskip

\cvevent{Contribution to LiSA static analyzer}{\cvtag{Java}\cvtag{Gradle}}{}{}

\faGithub\href{https://github.com/faccimatteo/StringGraphDomain-LiSA}{String graph domain LiSA}
%%\faGithub\href{https://github.com/lisa-analyzer/lisa}{LiSA}
\par

Contributed to LiSA static analyzer extending domain with String graphs.

\end{paracol}


\end{document}
